%% Template for Memo Series
%% Please modify the MEMO-data section below and
%% paste the report text into the space at the end of this file.
\documentclass[british]{article}
\usepackage{mathptmx}
\usepackage[T1]{fontenc}
\usepackage[latin9]{inputenc}
\usepackage[letterpaper]{geometry}
\geometry{verbose,tmargin=3cm,bmargin=3cm,lmargin=3cm,rmargin=3cm}
\usepackage{fancyhdr}
\pagestyle{fancy}
\usepackage{graphicx}
\usepackage{setspace}
\usepackage{amsmath}
\usepackage{amssymb}
\usepackage{babel}
\usepackage{algpseudocode}
\usepackage{algorithm}
\usepackage{float}
\usepackage{blindtext}
\usepackage{array}
\onehalfspacing

\makeatletter

%%%%%%%%%%%%%%%%%%%%%%%%%%%%%% LyX specific LaTeX commands.
\newcommand{\lyxline}[1][1pt]{%
  \par\noindent%
  \rule[.5ex]{\linewidth}{#1}\par}
  
  \@ifundefined{showcaptionsetup}{}{
 \PassOptionsToPackage{caption=false}{subfig}}
\usepackage{subfig}
\makeatother

\makeatother

\usepackage{babel}

\begin{document}
%% MEMO-data section to define report name, authors, contacts, date, number, etc.
%% Please modify it as appropriate.
\def\MEMOtitle{Automated translation of asynchronous concepts to Signal Transition Graphs}
\def\MEMOauthors{Jonathan Beaumont}
\def\MEMOcontacts{j.r.beaumont@ncl.ac.uk}
\def\MEMOyear{2016}
\def\MEMOdate{December \MEMOyear}
\def\MEMOnumber{NCL-EEE-MICRO-MEMO-\MEMOyear-013}
%% End od MEMO-data section

%% Page decoration options for the body of the memo
\pagenumbering{arabic}
\renewcommand{\headrulewidth}{0.4pt}
\renewcommand{\footrulewidth}{0.4pt}
\lhead{\MEMOauthors: \MEMOtitle}
\chead{}
\rhead{}
\lfoot{\MEMOnumber, Newcastle University}
\cfoot{}
\rfoot{\thepage}
\sloppy

\thispagestyle{empty}

\begin{center}
{\Large ~}{\footnotesize \vspace{-22mm}
}
\par\end{center}{\footnotesize \par}

{\Large \lyxline{\Large}}{\Large \par}

\begin{center}
%
\begin{minipage}[b][1\totalheight][t]{0.45\columnwidth}%
{\footnotesize \includegraphics[scale=0.14]{Images/Newcastle_Master_Col}}{\footnotesize \par}

\noindent {\footnotesize \smallskip{}
}{\footnotesize \par}

{\footnotesize Copyright~\copyright~\MEMOyear~Newcastle University}%
\end{minipage}%
\begin{minipage}[b][1\totalheight][t]{0.45\columnwidth}%
\noindent {\footnotesize $µ$Systems Research Group}\\
{\footnotesize School of Electrical and Electronic Engineering}\\
{\footnotesize Merz Court}\\
{\footnotesize Newcastle University}\\
{\footnotesize Newcastle upon Tyne, NE1 7RU, UK}{\footnotesize \par}

\noindent {\footnotesize \medskip{}
}{\footnotesize \par}

\noindent \texttt{\footnotesize http://async.org.uk/}%
\end{minipage}
\par\end{center}

\lyxline{\normalsize}

\vspace{15mm}

\newcommand{\noun}[1]{\textsc{#1}}

\begin{center}
\textbf{\huge \MEMOtitle}
\par\end{center}{\huge \par}

\begin{center}
\bigskip{}
{\Large \MEMOauthors}
\par\end{center}{\Large \par}

\begin{center}
\texttt{\small \MEMOcontacts}
\par\end{center}{\small \par}

\begin{center}
\bigskip{}
{\large \MEMOnumber}
\par\end{center}{\large \par}

\begin{center}
{\large \MEMOdate}
\par\end{center}{\large \par}

\vspace{15mm}

\begin{abstract}
Asynchronous circuits are becoming increasingly important in
system design for Internet-of-Things, where they orchestrate
the interface between big synchronous computation components
and the analogue environment, which is inherently asynchronous
and has high uncertainty with respect to power supply,
temperature and long-term ageing effects.
However, wide adoption of asynchronous circuits by industrial users is
hindered by a steep learning curve for asynchronous control models,
such as Signal Transition Graphs, that are developed by the academic
community for specification, verification and synthesis of
asynchronous circuits.

Previously, we have introduced a novel high-level description language
for asynchronous circuits, which is based on behavioural
\textit{concepts} -- high-level descriptions of asynchronous circuit
requirements, that can be shared, reused and extended by users. 
In this paper we will discuss more examples using concepts, and an algorithm to
automatically translate these to Signal Transition Graphs for further processing
by conventional asynchronous and synchronous EDA tools, such as \noun{Petrify}
and \noun{Mpsat}. Our aim is to simplify the process of capturing system
requirements in the form of a formal specification, and to promote behavioural
concepts as a means for design reuse. The proposed design flow is fully
automated in open-source toolsuite \noun{Workcraft}.
\end{abstract}
\newpage{}

\thispagestyle{fancy}


%\section{Introduction}

\include{Concepts}

%The memo goes here followed by the main sections and conclusions.
\end{document}
